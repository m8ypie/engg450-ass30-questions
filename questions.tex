%%%%%%%%%%%%%%%%%%%%%%%%%%%%%%%%%%%%%%%%%%%%%%%%%%%%%%%%%%%%%%%%%%%%%
% LaTeX Template: Project Titlepage Modified (v 0.1) by rcx
%
% Original Source: http://www.howtotex.com
% Date: February 2014
%
% This is a title page template which be used for articles & reports.
%
% This is the modified version of the original Latex template from
% aforementioned website.
%
%%%%%%%%%%%%%%%%%%%%%%%%%%%%%%%%%%%%%%%%%%%%%%%%%%%%%%%%%%%%%%%%%%%%%%

\documentclass[12pt]{article}
  % \renewcommand*\descriptionlabel[1]{\hspace\leftmargin$#1$}
  \usepackage[a4paper]{geometry}
  \usepackage[myheadings]{fullpage}
  \usepackage{fancyhdr}
  \usepackage{lastpage}
  \usepackage[shortlabels]{enumerate}
  \usepackage{graphicx, wrapfig, subcaption, setspace, booktabs, amsmath}
  \usepackage[T1]{fontenc}
  \usepackage{wrapfig}
  \usepackage[font=small, labelfont=bf]{caption}
  \usepackage{fourier}
  \usepackage{float}
  \usepackage[protrusion=true, expansion=true]{microtype}
  \usepackage[english]{babel}
  \usepackage{sectsty}
  \usepackage{url, lipsum}
  % \setlength{\parindent}{0pt}
  \pagenumbering{gobble}
\begin{document}
\section*{}
Ten components were tested for 500 hours, each within prescribed operating conditions.
 Component 1 failed after 30 hours; component 2 failed after 85 hours; component 3 failed after 220 hours;
 and component 4 failed after 435 hours. Determine the overall failure rate ($\lambda$) for the system.
 
\newpage
\section*{}
Assume that there is a requirement for a new systm with a specified performance capability and a reliability of 70\%.
In response to an \"invitation to bid\", three supplier configurations have been proposed and are reflected in Figure 28.
The component reliability factors are noted in the following table:

% Table generated by Excel2LaTeX from sheet 'Sheet1'
\begin{table}[htbp]
  \centering
    \begin{tabular}{cc|cc|cc}
    \toprule
    Component & Reliability & Component & Reliability & Component & Reliability \\
    \midrule
    A     & 0.84  & G     & 0.87  & M     & 0.83 \\
    B     & 0.86  & H     & 0.88  & N     & 0.85 \\
    C     & 0.89  & I     & 0.89  & O     & 0.84 \\
    D     & 0.86  & J     & 0.86  & P     & 0.89 \\
    E     & 0.87  & K     & 0.85  & Q     & 0.89 \\
    F     & 0.82  & L     & 0.86  &       &  \\
    \bottomrule
    \end{tabular}%
  \label{tab:addlabel}%
\end{table}%

The overall cost associated with each of the supplier proposals is \$57,000 for Configuration A, \$39,000 for
Configuration B and \$42,000 for Configuration C.
\begin{enumerate}[(a)] % (a), (b), (c), ...
  \item Determine the system reliability for each of the three configurations.
  \item In evaluating the three alternatives, employing cost-effectiveness (CE) criteria, which configurations would you select as being preferred?
\end{enumerate}

\newpage
\section*{}
Describe each of the following, its purpose, application, and the information acquired:
\begin{enumerate}[(a)] % (a), (b), (c), ...
  \item Failure mode, effects, and criticality analysis (FMECA)
  \item Fault-tree analysis (FTA)
  \item Stress-strength analysis
  \item Reliability prediction
  \item Reliability growth analysis
\end{enumerate}

\end{document}